\section{B}
\paragraph{Step 1: Describe the algorithm} \mbox{} \\
We now describe an algorithm $A$ for solving the problem.
\begin{enumerate}
    \item The algorithm receives the real input $X$, i.e., a set of edges and their weights, a set of vertices $V$ and a maximum value $B$ which indicates the upper bound for the tree and its mirrored weight, and the random sequence $R$ which consist of exactly $k$ bits: $R = r_1, r_2, \ldots, r_k$.
    \item It is assumed that there are more bits than edges.
    \item Consider the first $m$ bits. Create a graph $G' \subseteq G$ where if bit $i$ is 1 then edge $i$ is part of $G'$.
    \item If $G'$ is not a spanning tree of $G$ answer is \verb=NO=.
    \item If the weight of $G'$ as well as its mirror is less than or equal $B$ answer is \verb=YES=.
    \item Otherwise answer is \verb=NO=.
\end{enumerate}

\paragraph{Step 2a: The conditions are met} \mbox{}
\begin{enumerate}
    \item Assume that the answer to the input $X$ is \verb=YES=.
    \item Then there is a spanning tree $T'$ satisfying \textsc{Mfmst}.
    \item Let $z_1, z_2,\ldots,z_n$ be the sequence of edges in $T'$.
    \item Construct a bit sequence $R$ where $z_i = 1$ if $i$ is an edge in $T'$ and $z_i = 0$ otherwise.
    \item When $A$ receives $R^*$, construct exactly the spanning tree $T'$ satisfying the \textsc{Mfmst} problem. Then answer \verb=YES=. 
    \item Altogether there is a sequence of length $m$ that will give \verb=YES=. The probability of randomly creating it is positive since it is drawn uniformly at random from a finite set. 
\end{enumerate}

\paragraph{Step 2b: The conditions are not met} \mbox{}
\begin{enumerate}
    \item Assume that the answer to the input $X$ is \verb=NO=.
    \item Then there is no spanning tree $T'$ satisfying \textsc{Mfmst}.
    \item If $G'$ created from $R^*$ is not a tree, then the algorithm answers \verb=N=.
    \item Otherwise it creates the spanning tree specified by $R^*$.
    \item Its weight and the weight of the mirrored edges are compared to $B$.
    \item As either the spanning tree or the summed weight of the mirrored edges are more than $B$, the answer is \verb=NO=. 
\end{enumerate}

\paragraph{Step 3: Running time} \mbox{} \\
We need to show that $A$ is $p$-bounded for some polynomial $p$.
\begin{enumerate}
    \item There are $m$ edges. Space: $O(m)$.
    \item All edges are either marked or not marked. Time: $O(m)$.
    \item The weights of the marked edges are summed, and the weight of the mirrored edges are summed. It is assumed that the identification of a mirrored edge is an atomic action. Time: $O(m)$.
    \item The maximum of the accumulated weights are compared to $B$ and the answer is returned. Time: $O(1)$.
\end{enumerate}
All together the running time is $O(m)$ and space $O(m)$. Choosing a linear polynomial $p(m) = cm$ for some constant $c$ will do. 
\newline \newline
This completes the proof.