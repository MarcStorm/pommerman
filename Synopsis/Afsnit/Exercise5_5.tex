\section*{Exercise 5.5}
To show that \textsc{3-Sat-With-Imperfection} is $\mathcal{NP}$-complete, we first need a suitable problem $P_c$. \textsc{3-Sat} is a problem that is known to be $\mathcal{NP}$-complete, so for this proof $P_c = \textsc{3-Sat}$.

Now we want to prove
\begin{align*}
	P_c &\leq_p \textsc{3-Sat-With-Imperfection} \Rightarrow \\
	\textsc{3-Sat} &\leq_p \textsc{3-Sat-With-Imperfection}
\end{align*}

This input to the transformation is an instance of \textsc{3-Sat}, that is a set of clauses $C = \{ c_1, \ldots, c_l \}$, where every clause contains exactly 3 literals, or in other words their length is 3. The clauses are over $m$ boolean variables $\{ x_1, \ldots, x_m \}$. The transformation of $C$ will simply copy all the clauses of $C$ and add two new clauses, $c_{l+1}, c_{l+2}$. Furthermore will there be introduced three helper variables $y_j, 1 \leq j \leq 3$, which only will be used in the new clauses. Satisfiability is maintained, as a satisfying assignment for $c_j$ satisfy all $c'_{j}$ and vice versa. Now let $\textit{\textbf{X}} = (X,C)$ be the \textsc{3-Sat} and $\textit{\textbf{X'}} = (X',C') = T(\textit{\textbf{X}})$ be the resulting \textsc{3-Sat-With-Imperfection} instance.

Let $c_j = z_1 \lor z_2 \lor z_3$. We introduce three new helper variables $y_{j,1}, y_{j,2}, y_{j,3}$ and two new clauses:
\begin{align*}
	c'_{j} &= z_1 \lor z_2 \lor z_3 \\
	c'_{l+1} &= y_{j,1} \lor y_{j,2} \lor y_{j,3} \\
	c'_{l+2} &= \overline{y_{j,1}} \lor \overline{y_{j,2}} \lor \overline{y_{j,3}}
\end{align*}

For all the clauses in $C$, only two new clauses and three helper variables are created. In other word the running time of the transformation is $O(l)$.

Consider a truth assignment to \textit{\textbf{X}} which satisfies
\begin{align*}
	c_j = z_1 \lor z_2 \lor z_3
\end{align*}
Now we show that there is a truth assignment to \textit{\textbf{X'}} which satisfy all $c'_{j,i}$. All of the literals in $c_j$ are true and we claim that this implies that at least one literal in $C'$ is false and will satisfy $C'$.
\begin{align*}
	c'_{j} &= \textcolor{myred}{z_1} \lor \textcolor{myred}{z_2} \lor \textcolor{myred}{z_3} \\
	c'_{l+1} &= y_{j,1} \lor y_{j,2} \lor y_{j,3} \\
	c'_{l+2} &= \overline{y_{j,1}} \lor \overline{y_{j,2}} \lor \overline{y_{j,3}}
\end{align*}
\textcolor{myred}{Red} literals true by assumption, \textcolor{mygreen}{green} literals set to true.
Let's now consider how the above looks if we set $y_{j,1} = y_{j,2} = \text{true}$ and $y_{j,3} = \text{false}$
\begin{align*}
	c'_{j} &= \textcolor{myred}{z_1} \lor \textcolor{myred}{z_2} \lor \textcolor{myred}{z_3} \\
	c'_{l+1} &= \textcolor{mygreen}{y_{j,1}} \lor \textcolor{mygreen}{y_{j,2}} \lor y_{j,3} \\
	c'_{l+2} &= \overline{y_{j,1}} \lor \overline{y_{j,2}} \lor \textcolor{mygreen}{\overline{y_{j,3}}}
\end{align*}
$C'$ is still satisfied as at least one clause in which at most 2 literals are true.

Now consider a truth assignment to $X'$ that satisfy all clauses $c'_{j,i}$. We claim that this implies that at least one literal $z_1, z_2, z_3$ is true, whence $c_j$ is satisfied. For a contradiction we now assume that
\begin{align*}
	z_1 = z_2 = z_3 = \text{false}
\end{align*}
If we look at how this assumption affects the clauses 
\begin{align*}
	c'_{j} &= \textcolor{myred}{z_1} \lor \textcolor{myred}{z_2} \lor \textcolor{myred}{z_3} \\
	c'_{l+1} &= y_{j,1} \lor y_{j,2} \lor y_{j,3} \\
	c'_{l+2} &= \overline{y_{j,1}} \lor \overline{y_{j,2}} \lor \overline{y_{j,3}}
\end{align*}
\textcolor{myred}{Red} literals false by assumption. No matter how truth values are assigned to $y_{j,1}, y_{j,2}, y_{j,3}$, $c'_j$ will not be satisfied, contrary to our assumption. Hence the assumption that $z_1, z_2, z_3$ are false is not correct.

We have now shown that an assignment satisfying $c_j$ can be extended to one satisfying all $c'_{j,i}$ and that an assignment satisfying all $c'_{j,i}$ satisfy $c_j$. This concludes the proof that \textsc{3-Sat-With-Imperfection} is $\mathcal{NP}$-complete.