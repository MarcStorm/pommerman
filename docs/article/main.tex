% Template for ICASSP-2010 paper; to be used with:
%          mlspconf.sty  - ICASSP/ICIP LaTeX style file adapted for MLSP, and
%          IEEEbib.bst - IEEE bibliography style file.
% --------------------------------------------------------------------------
\documentclass{article}
\usepackage{amsmath,graphicx,02460}
\usepackage{dirtytalk}
\usepackage{url}
\usepackage[super]{nth}



\toappear{02456 Deep Learning, DTU Compute, Autumn 2018}


% Example definitions.
% --------------------
\def\x{{\mathbf x}}
\def\L{{\cal L}}

% Title.
% ------
\title{Pommerman: Deep Reinforcement Learning}
%
% Single address.
% ---------------
\name{Alexander D. Juhl, Marc S. Larsen, Marcus S. Hansen, Mark B. Jensen, Mathias P.~V. Mortensen \\ s134073, s144452, s144438, s144474, s144450}
%\name{Mark B. Jensen (s144474), Mathias P.~V. Mortensen (s144450)}
%\address{}
%
% For example:
% ------------
\address{Technical University of Denmark - DTU\\
	Department of Applied Mathematics and Computer Science\\
	Richard Petersens Plads, Building 324, DK-2800 Kgs. Lyngby}
%
% Two addresses (uncomment and modify for two-address case).
% ----------------------------------------------------------
%\twoauthors
%  {Alexander D. Juhl (s134073), Marc S. Larsen (s144452)}%
%	{School A-B\\
%	Department A-B\\
%	Address A-B}
 %   {Marcus S. Hansen (s144438), Mark B. Jensen (s144474), Mathias P.~V. Mortensen (s144450)}
%	{School C-D\\
%	Department C-D\\
%	Address C-D}
%
\begin{document}
%\ninept

\maketitle

\begin{onecolabstract}
\noindent 
\end{onecolabstract}
\section{Introduction}
\label{sec:intro}

% Reinforcement learning has previously been used to solve a number of different games such as Chess\footnote{\href{https://arxiv.org/pdf/1712.01815.pdf}{Mastering Chess and Shogi by Self-Play with a General Reinforcement Learning Algorithm}} and Go\footnote{\href{https://www.nature.com/articles/nature24270.epdf}{Mastering the game of Go without human knowledge}}, both of which are zero sum games. Reinforcement learning has excelled at playing Go, where the number of possible moves are so many that a traditional heuristic is not sufficient to solve the problem successfully. These games only contain two players, whereas a game like Dota II is 5 vs. 5 game. That makes it a multi-agent environment and a non-zero sum game, which introduces additional challenges for machine learning.\footnote{\href{https://blog.openai.com/openai-five/}{OpenAI Five}}

% Motivation
% The motivation for researching state of the art techniques for solving these problems is intriguing. Although much research has been done in the past years we have not yet seen revolutionary approaches, making headline such as AlphaZero or AlphaGo. We as a group would like to explore these state of the art approaches on the game Pommerman to evaluate how far machine learning has come, solving non-zero sum games in single- and multi-agent environments. We do not strafe for ground breaking research within the topic.


Accomplishing tasks with infinitely meaningful variation is common in the real world and difficult to simulate. One such place where this can be simulated is in games. In this poster the game pommerman will be used to simulate such an environment. The field of reinforcement learning has achieved impressive results in beating games.\cite{mnih2015a}\cite{silver2016a} The goal is to explore different deep reinforcement learning methods in order to see if they can enable an agent to beat simple and random agents in pommerman.

\begin{figure}[htb]
    \centerline{\includegraphics[width=0.7\linewidth]{docs/article/inputs/4pommerview.jpg}}
    \caption{Typical games of pommerman.}
    \label{fig:pomIntro}
\end{figure}
%\section{Background}
% Basically knowledge needed to understand the article. Here could be things from the book if we deem it needed (maybe ask monday?)


\section{Related Work}
\label{sec:relatedwork}
% Related work is other articles and solutions. Can often be in introduction or background if we want.

% Questions to answer for each source
% - What problem do they solve?
% - How do they solve it?
% - How does it relate to Pommerman?

The literature show many different solutions for approaching single- and multi-agent nonstationary reinforcement learning problems like Pommerman.

In a single agent system has \cite{rwightman} been able to train an agent to solve the Pommerman problem in a FFA with a win rate of $95\%$. This has been achieved by utilising actor-critic algorithm together with spatial feature representation and a CNN based model. However this win rate is purely achieved by playing defensively and relies on the other so called simple agents from the Pommerman environment to kill themselves in the FFA.

An alternative to the actor-critic algorithm is the Q-learning algorithm, which has been used by \cite{tokic2010adaptive} to train an agent to play the original Bomberman game, which Pommmerman is a variation of. The focus of the research done by \cite{tokic2010adaptive} is to test how different methods for balancing the exploration versus exploitation dilemma in reinforcement learning.

% https://cs.gmu.edu/~eclab/papers/panait05cooperative.pdf
%\subsection{Cooperative Multi-Agent Learning: The State of the Art}

% https://arxiv.org/pdf/1706.02275.pdf
%\subsection{Multi-Agent Actor-Critic for Mixed Cooperative-Competitive Environments}
\section{Methods}
\label{sec:methods}
% FROM POSTER
% Given the different methods that can be used for training a reinforcement learning agent an environment has been set up where one can easily change the type of opponents within the Pommerman environment. Furthermore the ability to change between a value-function-based method such as Q-learning, and a policy search method realised with the gradient estimator REINFORCE has been implemented.\cite{sutton1998a} The structure of the environment can be seen in figure \ref{fig:uml}.

% Methods used in our solution, could here be network choice, algorithm, parameter-choice etc.
% REINFORCE paper reference
% Network?
% Why was the chosen methods used

% http://incompleteideas.net/book/bookdraft2017nov5.pdf
\subsection{Reinforcement Learning: An Introduction} % Monte carlo, n-step TD, REINFORCE, actor critic.

% Questions to answer:
% - Why was the method used?
% - How does it work?

\subsection{REINFORCE algorithm}
\label{sec:reinforce}

\subsection{Q-learning algorithm}
% mention bomberman article

\subsection{Convolutional Network}
\label{sec:conv}
The policy neural network model has been accomplished by three convolutional layers followed by two fully connected layers as seen in figure~\ref{fig:network}. To achieve regularisation of the data, batch normalisation has been utilised after each convolutional layer. The activation function is ReLU.

\begin{figure}[htb]
    \centerline{\includegraphics[width=1.0\linewidth]{docs/article/inputs/conv.pdf}}
    \caption{Layout of the neural network}\label{fig:network}
\end{figure}

Some feature engineering has been performed on the input data to reduce complexity (i.e. supplying agent position as sets of coordinates may introduce unnecessary complexity). The state is represented by a $11\times 11\times 4$ tensor in which the following information is conveyed:
\begin{itemize}
    \item A $11\times 11$ tensor conveying obstacle positions
    \item A $11\times 11$ tensor conveying the agent's position
    \item A $11\times 11$ tensor conveying enemy agent positions
    \item A $11\times 11$ tensor conveying \emph{danger zones}
\end{itemize}
The term \emph{danger zones} is referred to as positions in which a bomb will eventually detonate and cause an agent harm. The intention is to represent the state as straightforward as possible to increase learning rates.


\subsection{Exploration / Exploitation}
% The number of possible states in the pommerman environment is enormous
To address the inherent problem in reinforcement learning of exploration versus exploitation while training the agent, was an $\epsilon$-greedy strategy with a decay introduced. The decay is defined as a function $\gamma$ of time.\cite{nieuwdorp2017dare} Adding the $\gamma$ function somewhat allows for control of when the agent should start exploiting what it has learned. In figure~\ref{fig:gamma} are the values of $\epsilon$ illustrated, where it is decayed by two different $\gamma$ functions. In both cases have an initial value of $\epsilon=1$ been used, i.e. the first iteration is $100\%$ random actions.

\begin{figure}[htb]
    \begin{minipage}[b]{.48\linewidth}
        \centering
        \centerline{\includegraphics[width=\linewidth]{pommerman/plots/epsilon_8.pdf}}
        \centerline{(a) $\epsilon$ value where $\gamma(x=8)$.}\medskip
    \end{minipage}
    \hfill
    \begin{minipage}[b]{0.48\linewidth}
        \centering
        \centerline{\includegraphics[width=\linewidth]{pommerman/plots/epsilon_20.pdf}}
        \centerline{(b) $\epsilon$ value where $\gamma(x=20)$.}\medskip
    \end{minipage}
    \caption{$\gamma$ = $n / (i / (n / x) + n)$ where $i \in iterations$.}
    \label{fig:gamma}
\end{figure}

(a) is extending the exploration, such that the value of $\epsilon \approx 0$ when approximately $85\%$ of the iterations has been carried out. (b) on the other hand will almost exclusively exploit when approximately $60\%$ of the iterations has been carried out, that is the value of $\epsilon \approx 0$.

% Not completely sure the argument blow is valid as the two graphs have different slopes.
(a) was primarily introduced to maximise the exploration, while keeping the amount of iterations for training at a minimum, as it would require roughly $42\%$ more iterations to achieve the same quantity of exploration with (b).

Another technique that was added in order to try and motivate the agent to explore the map, was to introduce a so called \emph{stop agent}, that would simply perform the action \emph{Stop} all the time. Both the random and simple agents have a probability of blowing themselves up, which will reward our agent with a positive reward if it is the last one standing. If this positive reward is achieved by doing nothing, it could promote the actions that was taken in that iteration thereby increasing the probability for performing a single action. The idea behind having the other agents standing still is that our agent would only be able to achieve a positive reward by exploring the board and potentially blow up its opponents. If all agent are standing still for long enough the game will end in a tie every the agent are rewarded with a negative reward.

\subsection{Static Board}
\label{sec:static}
% Static board to reduce complexity
To reduce the complexity of the search space has the option for keeping the Pommerman board static been introduced for both the training and validation of the network. This is rather than using an new randomly generated board every time a training or validation is carried out as is the case in figure~\ref{fig:pomIntro}. By doing so reduces the number of different boards the agent has to learn to a single board and thereby reducing the total number of possible states tremendously.

\subsection{Reward Function}
% Simple +/-1 1
% Cite bomberman/our tweaked version
Functionality to change the reward function has also been introduced. Besides the standard reward function that comes with the Pommerman environment, where the agents receives a score of 1 for winning and -1 for loosing, some agents were trained with reward shaping. The general motivation for introducing the reward shaping was to encourage the agent to avoid getting stuck in a local minimum. One of the reward functions used for this was the same as was used by \cite{kormelink2018exploration} and can be seen in table~\ref{tab:rew}. 

\begin{table}[htb]
    \centerline{
        \begin{tabular}{|l|r|}
            \hline
            Action                  & Reward   \\ 
            \hline
            Blow up opponent 		& $100$	\\
            Blow up wall  			& $30$  \\
            Perform action			& $-1$	\\
            Perform illegal action	& $-2$	\\
            Die  					& $-300$\\
            \hline
        \end{tabular}
    }
    \caption{A table with an example of reward shaping.}\label{tab:rew}
\end{table}

This specific reward function was introduced in an attempt to encourage the agent to explore the map by rewarding the agent for blowing up wall and giving it negative reward for moving and performing illegal actions, like walking into a wall. In other words if the agent wish to receive a total positive reward it has to blow up opponents or walls.
\section{Results}
The results were achieved when training against two types of agents, random and simple agents. In figure \ref{fig:resultsrandom} the validation and training error of the agent trained against random agents is displayed.

\begin{figure}[htb]
    \centerline{\includegraphics[width=1.0\linewidth]{pommerman/plots/random_train_val.pdf}}
    \caption{Validation and training error against random agents.}\label{fig:resultsrandom}
\end{figure}

In figure \ref{fig:resultssimple} the validation and training error of the agent trained against simple agents is displayed. The results turned out to be independent of the reward function, furthermore the results were nearly identical regardless of using REINFORCE or Q-learning.

\begin{figure}[htb]
    \centerline{\includegraphics[width=1.0\linewidth]{pommerman/plots/train_val.pdf}}
    \caption{Validation and training error against simple agents.}\label{fig:resultssimple}
\end{figure}

A lot of different agents were made and tested, including different hyper parameters and static board training. Each configuration was run for at least 150.000 games. However, despite these different configurations all networks ended up converging towards picking one action all the time, as is plotted in figure \ref{fig:act}. 

\begin{figure}[htb]
    \centerline{\includegraphics[width=1.0\linewidth]{pommerman/plots/a_probs.pdf}}
    \caption{Distribution of action probabilities.}\label{fig:act}
\end{figure}







% Sometimes mentioned as experiments/results. Maybe experiments sounds better for us xD
% Show that the agent was able to learn playing against 3 random agents
% Show graphs for result for playing against 3 simple agents (-1 constantly, easy graph)

\subsection{Performance}
% As in iterations, training time etc.
% Mention the number of iterations 

% Result for every method used should be stated
\section{Discussion}
\label{sec:discussion}
This section will discuss potential issues with the implemented methods that have been utilised during the training and validation of the agents.

\subsection{Reward Function}
% Make reference to book that it is a good idea to use -1,1 reward function
When making reward shaping is it of uttermost importance to keep the final goal in mind. Do we wish to learn the agent to play blow up walls and opponents or do we wish to learn it to play Pommerman? In this case, we obliviously want it to win the game of Pommerman and therefore \emph{\say{we must provide rewards to it in such a way that in maximizing them the agent will also achieve our goals. It is thus critical that the rewards we set up
truly indicate what we want accomplished.}}\cite{sutton1998a} By rewarding the agent for completing subgoals like blowing up walls and agents, we do not indicate our true goal. Rewarding the agent for these subgoals could lead to that the agent would blow up walls and opponents even at the cost of losing the game. Sticking to the reward function of $-1$ for losing and $1$ for winning, we tell the agent the we want it to win but not how it should be achieved.\cite{sutton1998a}

\subsection{Lack of Training}
% - Complexity
%   - It takes a lot of correct random actions to get a reward of 1 by killing the enemies
% - Exploration / Exploitation
It is evident from the literature that all the agents that do well have been trained for millions of times. An example of this is \cite{rwightman}'s agent, which has been trained for the same Pommerman environment for approximately 70 millions games. Another one is \cite{kormelink2018exploration} which has been trained for a million games, but in a $7 \times 7$ static classic Bomberman environment rather than a $11 \times 11$ ever changing environment. All of our trained setups showed the same sign of early convergence towards a single action. Our hypothesis is that the setups simple have not had enough time for training and a lot more training would yield for more exploration in the massive search space. This could result in that the agent would be able to take more meaningful actions rather than just sticking to a single action. The reasoning for this hypothesis is that it the amount of correct random actions the agent has to take in order to blow its way out of its initial position to reach the three other agents and to kill them in the process to be awarded with a reward of 1 is very small. Furthermore are the other agents reacting to our agent's movement in an attempt to stay alive and beat our agent. It is quite possible that the search space is simply too large for the agent to ever reach this state of actively beating the three other agents. It seems more likely that the agent could learn to avoid the simple agents and their bombs, as the simple agents are more likely to reach our agent before it is able to learn to place bombs and not killing itself, much like the behaviour of the agent that \cite{rwightman} trained. However, this will require a lot more training in order to verify whether this is true or not for our agent.

\subsection{Networks}

\subsection{Algorithms}
% Maybe talk about the problems with reinforce in the context of Pommerman. 
\section{Future Work}
% "Derp train more". Could be in discussion maybe?
% Actor critic
\section{Conclusion}
% Maybe not necessary if we have both results and discussion.

% Below is an example of how to insert images. Delete the ``\vspace'' line,
% uncomment the preceding line ``\centerline...'' and replace ``imageX.ps''
% with a suitable PostScript file name.
% -------------------------------------------------------------------------
%\begin{figure}[htb]

%\begin{minipage}[b]{1.0\linewidth}
%  \centering
%  \centerline{\includegraphics[width=8.5cm]{image1}}
%  \vspace{2.0cm}
%  \centerline{(a) Result 1}\medskip
%\end{minipage}
%
%\begin{minipage}[b]{.48\linewidth}
%  \centering
%  \centerline{\includegraphics[width=4.0cm]{image3}}
%  \vspace{1.5cm}
%  \centerline{(b) Results 3}\medskip
%\end{minipage}
%\hfill
%\begin{minipage}[b]{0.48\linewidth}
%  \centering
%  \centerline{\includegraphics[width=4.0cm]{image4}}
%  \vspace{1.5cm}
%  \centerline{(c) Result 4}\medskip
%\end{minipage}
%
%\caption{Example of placing a figure with experimental results.}
%\label{fig:res}
%
%\end{figure}

% To start a new column (but not a new page) and help balance the last-page
% column length use \vfill\pagebreak.
% -------------------------------------------------------------------------
\vfill
\pagebreak

% References should be produced using the bibtex program from suitable
% BiBTeX files (here: strings, refs, manuals). The IEEEbib.bst bibliography
% style file from IEEE produces unsorted bibliography list.
% -------------------------------------------------------------------------
\bibliographystyle{IEEEbib}
\bibliography{refs}

\vfill
\pagebreak
\section*{Appendix}
\appendix
The program has been structured such that one can easily swap different components, like the network, algorithm or opponents to train against. This modular structure of the program can be seen in figure~\ref{fig:uml}.
\begin{figure}[htb]
    \centerline{\includegraphics[width=0.8\linewidth]{docs/article/inputs/02456-UML.pdf}}
    \caption{UML of workflow.}\label{fig:uml}
\end{figure}


\end{document}
